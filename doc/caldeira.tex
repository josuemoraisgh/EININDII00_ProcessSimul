\documentclass{article}
\usepackage{amsmath}
\usepackage{graphicx}

\title{Modelagem Matem\'atica de uma Caldeira a Vapor e Sistema de Esteira de Cavaco}
\author{}
\date{}

\begin{document}

\maketitle

\section{Introdu\c{c}\~ao}
A modelagem de uma caldeira a vapor e do sistema de esteira de cavaco é essencial para a compreensão do comportamento dinâmico de sistemas térmicos e para o projeto de estratégias de controle eficientes. Este relatório apresenta um modelo matemático baseado em balanços de massa e energia para descrever a dinâmica da caldeira a vapor, incluindo a vazão de entrada de água, vazão de saída de vapor, nível de líquido, pressão interna da caldeira e a relação com a taxa de queima de combustível e a velocidade da esteira de cavaco.

\section{Defini\c{c}\~ao das Vari\'aveis}
O sistema pode ser descrito pelas seguintes variáveis principais:
\begin{itemize}
    \item $Q_{agua}$: Vaz\~ao de água de alimentação (kg/s)
    \item $Q_{vapor}$: Vaz\~ao de vapor (kg/s)
    \item $H$: Nível de líquido na caldeira (m)
    \item $P_{vapor}$: Pressão do vapor na caldeira (Pa)
    \item $q_{calor}$: Quantidade de calor fornecida (W)
    \item $T$: Temperatura da água/vapor ($^{\circ}$C)
    \item $m_f$: Taxa de queima de combustível (kg/s)
    \item $V_{esteira}$: Velocidade da esteira de cavaco (m/s)
    \item $LCV$: Poder calorífico inferior do combustível (MJ/kg)
    \item $C$: Capacidade térmica do sistema (J/K)
    \item $k$: Constante de proporcionalidade para a velocidade da esteira de cavaco
    \item $h_v$: Entalpia do vapor (kJ/kg)
    \item $A$: Área da seção transversal da caldeira (m²)
\end{itemize}

\section{Parâmetros Físicos da Caldeira}
Para realizar a modelagem e simulação, utilizamos valores típicos de uma pequena caldeira:
\begin{itemize}
    \item Volume total da caldeira: 2 m³
    \item Capacidade térmica da caldeira ($C_p$): 4.2 kJ/kgK
    \item Pressão de operação: 10 bar
    \item Temperatura do vapor: 180°C
    \item Vazão de alimentação ($Q_{agua}$): 0.5 kg/s
    \item Vazão de vapor ($Q_{vapor}$): 0.5 kg/s
    \item Eficiência da caldeira ($\eta$): 85%
    \item Poder calorífico inferior do combustível ($LCV$): 42 MJ/kg
    \item Taxa de queima de combustível ($m_f$): 0.01 kg/s
    \item Massa de ar na combustão ($m_{ar}$): 0.15 kg/s
    \item Coeficiente de vazão do vapor ($K_v$): 0.1
\end{itemize}

\section{Valores Típicos para Simulação}

\subsection{Parâmetros Físicos da Caldeira}
\begin{itemize}
    \item \textbf{Capacidade de Produção de Vapor}: 500 kg/h a 2000 kg/h
    \item \textbf{Pressão de Operação}: 6 a 10 bar (~600 a 1000 kPa)
    \item \textbf{Temperatura do Vapor Saturado}: ~165°C a 185°C
    \item \textbf{Volume do Tanque da Caldeira}: 0.5 m³ a 2.0 m³
    \item \textbf{Área da seção transversal (A)}: ~0.5 m² a 1.5 m²
\end{itemize}

\subsection{Vazões e Capacidade Térmica}
\begin{itemize}
    \item \textbf{Vazão de água de alimentação ($Q_{in}$)}: ~0.15 a 0.55 kg/s
    \item \textbf{Vazão de vapor ($Q_{out}$)}: ~0.15 a 0.55 kg/s
    \item \textbf{Capacidade térmica efetiva ($C_p$)}: ~1000 kJ/(kg·K)
    \item \textbf{Entalpia do vapor ($h_v$)}: ~2700 kJ/kg
\end{itemize}

\subsection{Exemplo para Simulação}
Para uma caldeira de **1000 kg/h (1 ton/h) operando a 8 bar**, pode-se utilizar:
\begin{itemize}
    \item \textbf{$Q_{in}$}: 0.28 kg/s
    \item \textbf{$Q_{out}$}: 0.28 kg/s
    \item \textbf{$P$}: 800 kPa
    \item \textbf{$Q_{calor}$}: 810 kW
    \item \textbf{$h_v$}: 2770 kJ/kg
    \item \textbf{$C_p$}: 1000 kJ/(kg·K)
    \item \textbf{$A$}: 1.0 m²
\end{itemize}

\section{Modelagem Matem\'atica}

\subsection{Balan\c{c}o de Massa (Din\^amica do N\'ivel de L\'iquido)}
A variação do nível de líquido na caldeira é determinada pela diferença entre a vazão de entrada e a vazão de saída:

\begin{equation}
    \frac{dH}{dt} = \frac{1}{A} (Q_{agua} - Q_{vapor})
\end{equation}

No domínio de Laplace, essa equação pode ser representada como:

\begin{equation}
    H(s) = \frac{1}{A s} (Q_{agua}(s) - Q_{vapor}(s))
\end{equation}

\subsection{Balan\c{c}o de Energia (Din\^amica da Press\~ao e Produ\c{c}\~ao de Vapor)}
A pressão do vapor na caldeira depende da energia fornecida ao sistema:

\begin{equation}
    \frac{dP}{dt} = \frac{1}{C_p} (q_{calor} - Q_{vapor} \cdot h_v)
\end{equation}

No domínio de Laplace:

\begin{equation}
    P(s) = \frac{1}{C_p s} (q_{calor}(s) - h_v Q_{vapor}(s))
\end{equation}

\subsection{Rela\c{c}\~ao entre Calor e Temperatura da Fornalha}
A relação entre a produção de calor e a temperatura da fornalha é dada por:

\begin{equation}
    q_{calor} = \eta \cdot m_f \cdot LCV
\end{equation}

A temperatura da fornalha é aproximada por:

\begin{equation}
    T_f = \frac{q_{calor}}{m_{ar} \cdot C_p}
\end{equation}

\subsection{Vaz\~ao de Sa\'ida em Fun\c{c}\~ao da Press\~ao}
A relação entre a vazão de vapor e a pressão pode ser aproximada por:

\begin{equation}
    Q_{vapor}(s) = K_v P(s)
\end{equation}

Substituindo essa expressão na equação do nível:

\begin{equation}
    H(s) = \frac{1}{A s} \left( Q_{agua}(s) - K_v P(s) \right)
\end{equation}

\subsection{Modelo Combinado}
A equação combinada para o nível da caldeira, considerando a dinâmica de pressão e calor fornecido, pode ser expressa como:

\begin{equation}
    H(s) = \frac{1}{A s} \left( Q_{agua}(s) - K_v \cdot \frac{1}{C_p s} (q_{calor}(s) - h_v Q_{vapor}(s)) \right)
\end{equation}

E a equação da pressão permanece:

\begin{equation}
    P(s) = \frac{1}{C_p s} (q_{calor}(s) - h_v Q_{vapor}(s))
\end{equation}

\subsection{Função de Transferência para a Esteira de Cavaco}
A taxa de queima de combustível \( m_f \) pode influenciar a velocidade da esteira de cavaco \( V_{esteira} \) devido ao calor gerado. A função de transferência que relaciona \( m_f \) e \( V_{esteira} \) é dada por:

\[
G(s) = \frac{V_{esteira}(s)}{m_f(s)} = \frac{k \cdot \eta \cdot LCV}{C \cdot (s + \alpha)}
\]

Onde:
\begin{itemize}
    \item \( k \) é a constante de proporcionalidade para a velocidade da esteira,
    \item \( \eta \) é a eficiência da caldeira,
    \item \( LCV \) é o poder calorífico inferior do combustível,
    \item \( C \) é a capacidade térmica do sistema,
    \item \( \alpha \) é uma constante relacionada à dinâmica térmica do sistema.
\end{itemize}

\subsection{Valores Sugeridos para Simulação}
Para simulação dessa função de transferência, os seguintes valores podem ser utilizados:
\begin{itemize}
    \item \( \eta = 0.85 \) (85\% de eficiência),
    \item \( LCV = 42 \, \text{MJ/kg} \),
    \item \( C = 1000 \, \text{kJ/K} \),
    \item \( k = 0.5 \) (constante de proporcionalidade para a velocidade da esteira),
    \item \( \alpha = 0.1 \).
\end{itemize}

Com esses valores, a função de transferência para a esteira de cavaco pode ser utilizada para simular a resposta do sistema em relação à variação na taxa de queima de combustível.

\section{Introdu\c{c}\~ao}
A modelagem de uma caldeira a vapor e do sistema de esteira de cavaco \'e essencial para a compreens\~ao do comportamento din\^amico de sistemas t\'ermicos e para o projeto de estrat\'egias de controle eficientes. Este relat\'orio apresenta um modelo matem\'atico baseado em balan\c{c}os de massa e energia para descrever a din\^amica da caldeira a vapor, incluindo a vaz\~ao de entrada de \'agua, vaz\~ao de sa\'ida de vapor, n\'ivel de l\'iquido, press\~ao interna da caldeira e a rela\c{c}\~ao com a taxa de queima de combust\'ivel, a velocidade da esteira de cavaco e a velocidade do ventilador de ar.

\section{Defini\c{c}\~ao das Vari\'aveis}
O sistema pode ser descrito pelas seguintes vari\'aveis principais:
\begin{itemize}
    \item $Q_{agua}$: Vaz\~ao de \'agua de alimenta\c{c}\~ao (kg/s)
    \item $Q_{vapor}$: Vaz\~ao de vapor (kg/s)
    \item $H$: N\'ivel de l\'iquido na caldeira (m)
    \item $P$: Press\~ao do vapor na caldeira (Pa)
    \item $q_{calor}$: Quantidade de calor fornecida (W)
    \item $T$: Temperatura da \'agua/vapor ($^{\circ}$C)
    \item $m_f$: Taxa de queima de combust\'ivel (kg/s)
    \item $V_{esteira}$: Velocidade da esteira de cavaco (m/s)
    \item $V_{vent}$: Velocidade do ventilador de ar (m/s)
    \item $LCV$: Poder calor\'ifico inferior do combust\'ivel (MJ/kg)
    \item $C$: Capacidade t\'ermica do sistema (J/K)
    \item $k$: Constante de proporcionalidade para a velocidade da esteira de cavaco
    \item $h_v$: Entalpia do vapor (kJ/kg)
    \item $A$: \'Area da se\c{c}\~ao transversal da caldeira (m\textsuperscript{2})
    \item $\lambda$: Rela\c{c}\~ao ar/combust\'ivel (kg ar / kg combust\'ivel)
\end{itemize}

\section{Fun\c{c}\~oes de Transfer\^encia}
A taxa de queima de combust\'ivel ($m_f$) pode ser influenciada tanto pela velocidade da esteira de cavaco ($V_{esteira}$) quanto pela velocidade do ventilador de ar ($V_{vent}$), afetando a quantidade de calor gerada na combust\~ao.

A rela\c{c}\~ao entre a velocidade da esteira e a taxa de queima \'e dada por:
\begin{equation}
    G_{esteira}(s) = \frac{m_f(s)}{V_{esteira}(s)} = \frac{k_1}{s + \alpha_1}
\end{equation}

A rela\c{c}\~ao entre a velocidade do ventilador e a combust\~ao \'e:
\begin{equation}
    G_{vent}(s) = \frac{\lambda(s)}{V_{vent}(s)} = \frac{k_2}{s + \alpha_2}
\end{equation}

A produ\c{c}\~ao de calor pode ser descrita como:
\begin{equation}
    q_{calor}(s) = \eta \cdot m_f(s) \cdot LCV
\end{equation}

Substituindo $m_f(s)$:
\begin{equation}
    q_{calor}(s) = \eta \cdot \frac{k_1}{s + \alpha_1} V_{esteira}(s) \cdot LCV
\end{equation}

E a rela\c{c}\~ao entre o ventilador e a combust\~ao pode ser usada para estimar a mistura ideal de ar/combust\'ivel.

\section{Valores Sugeridos para Simula\c{c}\~ao}
Para simula\c{c}\~ao dessas fun\c{c}\~oes de transfer\^encia, podem ser utilizados os seguintes valores:
\begin{itemize}
    \item $\eta = 0.85$ (85\% de efici\^encia)
    \item $LCV = 42$ MJ/kg
    \item $C = 1000$ kJ/K
    \item $k_1 = 0.5$, $\alpha_1 = 0.1$
    \item $k_2 = 0.3$, $\alpha_2 = 0.05$
    \item Rela\c{c}\~ao ar/combust\'ivel $\lambda = 15$
\end{itemize}

\section{Conclus\~ao}
A modelagem matemática de uma caldeira e do sistema de esteira de cavaco permite entender sua dinâmica e projetar sistemas de controle eficientes. Com os valores apresentados, é possível simular diferentes cenários e otimizar a operação, tanto para a caldeira quanto para o controle da velocidade da esteira de cavaco.

\end{document}