\documentclass[12pt]{article}
\usepackage{amsmath,graphicx}
\usepackage{siunitx}
\usepackage{geometry}
\geometry{margin=2.5cm}

\title{Modelagem Din\^amica de Caldeira: Combust\'ivel, Ar e Temperatura da Fornalha}
\author{Josu\'e Morais}
\date{\today}

\begin{document}

\maketitle

\section*{1. Introdu\c{c}\~ao}
Este documento apresenta a modelagem din\^amica de uma caldeira considerando as entradas:
\begin{itemize}
    \item Abertura da v\'alvula de combust\'ivel l\'iquido (0 a 100\%)
    \item Abertura do damper de ar (0 a 100\%)
    \item Abertura da v\'alvula de \'agua (0 a 100\%)
    \item Press\~ao da bomba de \'agua (Pa)
\end{itemize}
E como sa\'idas:
\begin{itemize}
    \item Vaz\~ao de combust\'ivel (kg/s)
    \item Vaz\~ao de ar (kg/s)
    \item Vaz\~ao de \'agua (kg/s)
    \item Calor gerado na fornalha (W)
    \item Temperatura da fornalha (\si{\celsius})
    \item Press\~ao de vapor (Pa)
    \item N\'ivel da caldeira (m)
\end{itemize}

\section*{2. Valores T\'ipicos da Caldeira}
\begin{itemize}
    \item Vaz\~ao de combust\'ivel: \SI{0}{kg/s} a \SI{0.0227}{kg/s}
    \item Vaz\~ao de ar: \SI{0}{kg/s} a \SI{0.15}{kg/s}
    \item Vaz\~ao de vapor: \SI{0}{kg/s} a \SI{0.35}{kg/s}
    \item Vaz\~ao de \'agua: \SI{0.15}{kg/s} a \SI{0.55}{kg/s}
    \item Poder calor\'ifico inferior (LCV): \SI{42}{MJ/kg}
    \item Efici\^encia t\'ermica ideal (\(\eta_{\text{max}}\)): 0,85
    \item Calor espec\'ifico do ar (\(C_p\)): \SI{1000}{J/(kg\cdot\celsius)}
    \item Entalpia do vapor (\(h_v\)): \SI{2770}{kJ/kg}
    \item Raz\~ao estequiom\'etrica ideal (\(\lambda_{\text{ideal}}\)): 15
    \item Tempo de acomoda\c{c}\~ao da fornalha: \SI{5}{s}
\end{itemize}

\section*{3. Fun\c{c}\~oes de Transfer\^encia}
\subsection*{3.1 Vaz\~ao de Combust\'ivel}
\begin{equation}
\dot{m}_f(s) = \frac{0.0227}{2.5s + 1} \cdot u_f(s)
\end{equation}

\subsection*{3.2 Vaz\~ao de Ar}
\begin{equation}
\dot{m}_{ar}(s) = \frac{0.15}{1.25s + 1} \cdot u_a(s)
\end{equation}

\subsection*{3.3 Vaz\~ao de \'Agua}
\begin{equation}
\dot{m}_{agua}(s) = \frac{K_b}{\tau_b s + 1} \cdot u_w(s)
\end{equation}

Onde:
\begin{itemize}
    \item \( u_w(s) \): abertura da v\'alvula de \'agua
    \item \( K_b \): ganho proporcional \`a press\~ao da bomba (ex: \( K_b = 0.55 \) para press\~ao m\'axima)
    \item \( \tau_b \): constante de tempo do sistema hidr\'aulico
\end{itemize}

\textbf{Limites esperados:}
\begin{itemize}
    \item Vaz\~ao m\'axima de \'agua: \SI{0.55}{kg/s}
    \item Vaz\~ao m\'inima de \'agua: \SI{0.15}{kg/s}
\end{itemize}

\subsection*{3.4 Calor Gerado (com mistura estequiom\'etrica)}
\begin{equation}
q_{calor}(t) = \eta(\lambda(t)) \cdot \dot{m}_f(t) \cdot LCV
\end{equation}

\begin{equation}
\eta(\lambda) = \eta_{max} \cdot e^{-k(\lambda - \lambda_{ideal})^2}, \quad k = 0.05
\end{equation}

\subsection*{3.5 Din\^amica T\'ermica da Fornalha (1\textordfeminine ordem)}
\begin{equation}
q_{calor}(s) = \frac{1}{5s + 1} \cdot \left[ \eta(\lambda(s)) \cdot \dot{m}_f(s) \cdot LCV \right]
\end{equation}

\subsection*{3.6 Temperatura da Fornalha}
\begin{equation}
T(t) = \frac{q_{calor}(t)}{\dot{m}_{ar}(t) \cdot C_p}
\end{equation}

\subsection*{3.7 N\'ivel da Caldeira}
\begin{equation}
\frac{dH}{dt} = \frac{1}{A} (\dot{m}_{agua} - \dot{m}_{vapor})
\end{equation}

No dom\'inio de Laplace:
\begin{equation}
H(s) = \frac{1}{A s} (\dot{m}_{agua}(s) - \dot{m}_{vapor}(s))
\end{equation}

\section*{4. Limites Esperados}
\begin{itemize}
    \item Temperatura m\'axima: \SI{900}{\celsius} (mistura ideal)
    \item Temperatura m\'inima: \SI{25}{\celsius} (sem combust\'ivel ou ar)
    \item Calor m\'aximo: \SI{810390}{W}
    \item Calor m\'inimo: \SI{0}{W}
    \item Press\~ao de vapor m\'axima: \SI{1e6}{Pa} (10 bar)
    \item Press\~ao de vapor m\'inima: \SI{1e5}{Pa} (1 bar)
\end{itemize}

\section*{5. Observa\c{c}\~oes}
\begin{itemize}
    \item A din\^amica do calor e temperatura \'e essencial para representar o atraso f\'isico da queima e aquecimento.
    \item A efici\^encia depende fortemente da raz\~ao ar/combust\'ivel \(\lambda\), que deve se manter pr\'oxima de 15.
\end{itemize}

\section*{6. Press\~ao de Vapor em Fun\c{c}\~ao do Calor e da Vaz\~ao de Vapor}

A equa\c{c}\~ao diferencial que relaciona a varia\c{c}\~ao da press\~ao com o calor fornecido e a vaz\~ao de vapor \'e dada por:

\[
\frac{dP}{dt} = \frac{1}{C_p} \left( q_{\text{calor}} - \dot{m}_{\text{vapor}} \cdot h_v \right)
\]

No dom\'inio de Laplace, essa equa\c{c}\~ao se torna:

\[
P(s) = \frac{1}{C_p \cdot s} \left( q_{\text{calor}}(s) - h_v \cdot \dot{m}_{\text{vapor}}(s) \right)
\]

Substituindo as fun\c{c}\~oes de transfer\^encia para \( q_{\text{calor}}(s) \) e \( \dot{m}_{\text{vapor}}(s) \):

\[
q_{\text{calor}}(s) = \frac{810390}{5s + 1} \cdot u_f(s)
\]

\[
\dot{m}_{\text{vapor}}(s) = \frac{0.35}{2s + 1} \cdot u_v(s)
\]

Com isso, a fun\c{c}\~ao de transfer\^encia da press\~ao de vapor torna-se:

\[
P(s) = \frac{1}{10^6 \cdot s} \left( \frac{810390}{5s + 1} \cdot u_f(s) - \frac{969500}{2s + 1} \cdot u_v(s) \right)
\]

Onde:
\begin{itemize}
    \item \( C_p = 1.0 \times 10^6 \, \text{J/K} \)
    \item \( h_v = 2.77 \times 10^6 \, \text{J/kg} \)
    \item \( 0.35 \, \text{kg/s} \) \'e a m\'axima vaz\~ao de vapor
\end{itemize}

\end{document}